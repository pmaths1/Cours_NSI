\documentclass[]{scrartcl}

\usepackage{eurosym}
\usepackage{amsmath,amssymb,amsfonts}

\usepackage[T1]{fontenc}
\usepackage[french]{babel}
\usepackage[utf8]{inputenc}
\usepackage{aeguill}
\usepackage{hyperref}
\usepackage{fourier} % math & rm
\usepackage[scaled=0.875]{helvet} % ss
\usepackage[normalem]{ulem}
\usepackage{pifont}
\usepackage{minted}
\title{Q.C.M}
\author{ Pascal Malingrey, Pascal Seckinger, Patrick Th\'evenon}
\date{}

\begin{document}

\maketitle

\newpage

\section*{QCM - énoncé}
\begin{enumerate}
    \item Quel est le tag à utiliser pour créer un dessin dans une page web?
    \begin{enumerate}
      \item \verb?<figure> </figure>?
      \item \verb?<graph> </graph>?
      \item \verb?<img> </img>?
      \item \verb?<canvas> </canvas>?
    \end{enumerate}
    \item Quel est l'intérêt du \textbf{sprite} dans la programmation Web?
    \begin{enumerate}
      \item D'être une boisson rafraîchissante.
      \item De couper une page HTML en deux.
      \item De découper une image en plusieurs parties
      \item De charger une seule image formée de plusieurs morceaux afin de réduire la taille en octet.
    \end{enumerate}
    \item Pour centrer un \textbf{div} dont la classe est \textbf{mb} dans son conteneur on utilise :
    \begin{enumerate}
      \item \mintinline{html}{<center><div class='mb'>...</div></center>}
      \item \mintinline{css}{.mb{margin:auto; width:80%}}
      \item \mintinline{css}{.mb{text-align:center}}
      \item \mintinline{css}{.mb{padding:auto; width:80%}}
    \end{enumerate}
    \item On veut récupérer tous les éléments d'une liste
    
    \mintinline{html}{<ul id="serie"><li></li>...<li></li></ul>}
    
    en javascript, quelle est la bonne instruction parmi les suivantes ?
    \begin{enumerate}
      \item \mintinline{js}{liste = document.getElementById("serie")}
      \item \mintinline{js}{liste = document.getElementsById("li")}
      \item \mintinline{js}{liste = document.getElementById("serie").getElementsByTagName('li')}
      \item \mintinline{js}{liste = document.getElementsByTagName('li')}
    \end{enumerate}
    \item Pour parcourir un tableau associatif \textbf{somme} en javascript, on utilise :
    \begin{enumerate}
      \item \mintinline{js}{somme = {}; for (i in somme){...} }
      \item \mintinline{js}{somme = []; for (i in somme){...} }
      \item \mintinline{js}{somme = []; for (i in somme.length)){...} }
      \item \mintinline{js}{somme = {}; for (i in somme.length)){...} }
    \end{enumerate}
 \item On considère un \textbf{div} dont l'id est \textbf{donnees}. On récupère le noeud du DOM par la commande \mintinline{js}{corps = document.getElementById('donnees')}. Pour ajouter du texte à cet élément on va écrire :
    \begin{enumerate}
      \item \mintinline{js}{corps.HTML += "mon texte à ajouter"}
      \item \mintinline{js}{corps += "mon texte à ajouter"}
      \item \mintinline{js}{corps.innerHTML += "mon texte à ajouter"}
      \item \mintinline{js}{corps.outerHTML += "mon texte à ajouter"}
    \end{enumerate}
\end{enumerate}
\newpage
\section*{QCM - corrigé}

\begin{enumerate}
    \item Quel est le tag à utiliser pour créer un dessin dans une page web?
    \begin{enumerate}
        \item \verb?<figure> </figure>?\\ \textit{La balise n'existe pas.}
        \item \verb?<graph> </graph>?\\ \textit{La balise n'existe pas.}
        \item \verb?<img> </img>? \\ \textit{La balise  permet d'insérer une image (attention il s'agit en principe d'une balise auto-fermante).}
        \item \verb?<canvas> </canvas>?: \\ \textbf{bonne réponse}
    \end{enumerate}
    \item Quel est l'intérêt du \textbf{sprite} dans la programmation Web?
    \begin{enumerate}
        \item D'être une boisson rafraîchissante. \\ \textit{ la marque existe mais sans intérêt ici.}
        \item De couper une page HTML en deux. \\ \textit{plusieurs balises permettent de séparer la page en partie, par exemple div, aside...}
        \item De découper une image en plusieurs parties.\\ \textit{C'est un peu l'inverse.}
        \item De charger une seule image formée de plusieurs morceaux afin de réduire la taille en octet.\\ \textbf{bonne réponse}, \textit{l'intérêt est exprimé dans la question. On utilise après background-position pour délimiter la zone qui nous intéresse.}
    \end{enumerate}
    \item Pour centrer un \textbf{div} dont la classe est \textbf{mb} dans son conteneur on utilise :
    \begin{enumerate}
        \item \mintinline{html}{<center><div class='mb'>...</div></center>}\\
         \textbf{bonne réponse }, \textit{mais elle est dépréciée en  HTML4 (et XHTML 1) .}
        \item \mintinline{css}{.mb{margin:auto; width:80%}} \\
                 \textbf{bonne réponse }, la marge sera de 10% de par et d'autre.
                \item \mintinline{css}{.mb{text-align:center}} \\
                \textit{Ici c'est le texte à l'intérieur qui est centré.}
                \item \mintinline{css}{.mb{padding:auto; width:80%}} \\
                      \textit{Le bloc va prendre 80\% mais aucune indication n'est donnée sur son alignement. }
                    \end{enumerate}
                    \item On veut récupérer tous les éléments d'une liste        
                    \mintinline{html}{<ul id="serie"><li></li>...<li></li></ul>}
                    
                    en javascript, quelle est la bonne instruction parmi les suivantes ?
                    \begin{enumerate}
                        \item \mintinline{js}{liste = document.getElementById("serie")}\\
                        \textit{ On récupère le noeud ul.}
                        \item \mintinline{js}{liste = document.getElementsById("li")}\\ \textit{On récupère un élèment s'il existe dont l'id est li.}
                        \item \mintinline{js}{liste = document.getElementById("serie").getElementsByTagName('li')}\\ \textbf{bonne réponse}
                        \item \mintinline{js}{liste = document.getElementsByTagName('li')} \\ \textit{ On récupère tous les li (item d'une liste) de la page et pas uniquement ceux de la liste serie.}
                    \end{enumerate}
                    \item Pour parcourir un tableau associatif \textbf{somme} en javascript, on utilise :
                    \begin{enumerate}
                        \item \mintinline{js}{somme = {}; for (i in somme){...} } \\ \textbf{bonne réponse}
                        \item \mintinline{js}{somme = []; for (i in somme){...} } \\  \textit{Ici somme est tableau qui n'est pas associatif}
                        \item \mintinline{js}{somme = []; for (i in somme.length)){...} }\\  \textit{somme.length est la longueur de la liste, donc i in somme.length n'a pas de sens.}
                        \item \mintinline{js}{somme = {}; for (i in somme.length)){...} } \\ \textit{même remarque que précédemment.}
                    \end{enumerate}
                    \item On considère un \textbf{div} dont l'id est \textbf{donnees}. On récupère le noeud du DOM par la commande \mintinline{js}{corps = document.getElementById('donnees')}. Pour ajouter du texte à cet élément on va écrire :
                    \begin{enumerate}
                        \item \mintinline{js}{corps.HTML += "mon texte à ajouter"} \\\textit{ la méthode HTML n'existe pas.}
                        \item \mintinline{js}{corps += "mon texte à ajouter"} \\  \textit{corps est un objet qui n'est pas un string, donc on ne peut ajouter un string à cet objet.}
                        \item \mintinline{js}{corps.innerHTML += "mon texte à ajouter"} \textbf{bonne réponse}
                        \item \mintinline{js}{corps.outerHTML += "mon texte à ajouter"} \\ \textit{outerHTML prend en compte également la balise div, elle est devenue obsolète et n'est pas acceptée par tous les navigateurs.}
                    \end{enumerate}
                \end{enumerate}
            

\end{document}