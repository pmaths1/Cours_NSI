\PassOptionsToPackage{unicode=true}{hyperref} % options for packages loaded elsewhere
\PassOptionsToPackage{hyphens}{url}
%
\documentclass[10pt,french,A4]{article}
\usepackage{mathrsfs}
\usepackage{eurosym}
\usepackage{amsmath,amssymb,amsfonts}

\usepackage[T1]{fontenc}
\usepackage[french]{babel}

\usepackage[utf8]{inputenc}
\usepackage{aeguill}
\usepackage[colorlinks=true,urlcolor=blue]{hyperref}
\usepackage{array}
\makeatletter
%%%%%%%%%%%%%%%%%%%%%%%%%%%%%% Textclass specific LaTeX commands.
\usepackage{framed}
\usepackage[framed]{ntheorem}
\newframedtheorem{theoreme}{Théorème}

%definition
\def\theoremframecommand#1{\vrule\hspace{6pt}\hbox{#1}}
\setlength\theorempreskipamount{0pt}%
\setlength\theorempostskipamount{0pt}%
\newshadedtheorem{definition}{\protect\definitionname}
\theoremprework{%
    \setlength\theorempreskipamount{\topsep}%
    \setlength\theorempostskipamount{\topsep}%
}
\theoremstyle{plain}
\theorembodyfont{\upshape}
\newtheorem*{remarque}{Remarque}
\newframedtheorem{propriete}{Propriété}
\newtheorem*{proof}{Preuve}
\newtheorem*{exemple}{Exemple}
\newtheorem{exercice}{Exercice}

% Définition des environnements pour les théorèmes, proposition, etc...

\providecommand{\definitionname}{Définition}
\providecommand{\definitionname}{exemple}
\providecommand{\definitionname}{preuve}
\providecommand{\definitionname}{remarque}
\providecommand{\definitionname}{propriete}
%\providecommand{\theoremname}{Théorème}
%%%%%%%%%%%%%%%%%%%%%%%%%%%%%% User specified LaTeX commands.

\usepackage{fourier} % math & rm
\usepackage[scaled=0.875]{helvet} % ss
\usepackage[normalem]{ulem}
\usepackage{pifont}
\usepackage[tikz]{bclogo}
\usetikzlibrary{positioning,shapes,decorations}
\usepackage{wrapfig}
\usepackage{mathrsfs}
\usetikzlibrary{arrows}
\usepackage{multicol}

\setlength{\columnseprule}{0.5pt}
\setlength{\columnsep}{20pt}
\usepackage{smartdiagram}
\usepackage[paperwidth=210mm,paperheight=297mm]{geometry}
\geometry{verbose,tmargin=2cm,bmargin=15mm,lmargin=2cm,rmargin=2cm,headheight=1cm,headsep=5mm,footskip=5mm}
\usepackage{fancyhdr}
\usepackage{hyperref}
\hypersetup{
            pdftitle={Somme de dés},
            pdfauthor={Pascal Malingrey, Pascal Seckinger, Patrick Thévenon},
            pdfkeywords={NSI, HTML, CSS, Javascript},
            pdfborder={0 0 0},
            pdfsubject={Lancer de dés et javascript},
            breaklinks=true}
\urlstyle{same}  % don't use monospace font for urls
\usepackage{xcolor}
\usepackage[tikz]{bclogo}
\usepackage{wrapfig}
\usepackage{framed}
\usepackage{algorithm}
\usepackage{algpseudocode}%MISE EN PAGE

\usepackage{listings,lipsum,listingsutf8}
\definecolor{codegreen}{rgb}{0.1,0.47,0.1}
\definecolor{fondjaune}{rgb}{0.99, 0.7,0.8}
\definecolor{couleurdef}{rgb}{0.99, 0.8, 0.87}
\definecolor{codegray}{rgb}{0.5,0.5,0.5}
\definecolor{codepurple}{rgb}{0.58,0,0.82}
\definecolor{backcolour}{rgb}{0.95,0.95,0.92}

\newenvironment{code}[1]{%
    \begin{bclogo}[couleur=backcolour, couleurTexte=black ,couleurBord=blue ,couleurBarre=black, ombre=false,epBord=0.9,logo=\#,arrondi=0.1]{{\bfseries #1}}%
    }%
    {%
    \end{bclogo}
}%
%---------------------------- version élève - prof
\newboolean{Prof}

\newcommand{\cacher}[1]{
    \ifthenelse{\boolean{Prof}} % si « Professeur » est vrai,
    {#1} %les mots cachés sont en gras
    {\dashuline{\phantom{#1}}} % (else) sinon les mots sont remplacés par une ligne sur laquelle l'élève peut écrire. 
}
\newcommand{\cacherb}[1]{
    \ifthenelse{\boolean{Prof}} % si « Professeur » est vrai,
    {#1} %les mots cachés sont en gras
    {\phantom{#1}} % (else) sinon les mots sont remplacés par une ligne sur laquelle l'élève peut écrire. 
}
%-------------------------------------------------------
\usepackage{minted}
\title{Mini-projet :
\\
Statistiques sur la somme de dés}
\author{Pascal Malingrey, Pascal Seckinger, Patrick Thévenon}
\date{}
%------------------------------------------------------------
\usepackage{fancyhdr}
\pagestyle{fancy}
\lhead{\Large Projet: web}

\setlength{\parindent}{0mm}

\AtBeginDocument{ % Nécessaire à cause de babel
\renewcommand\labelitemi{\textbullet}
\renewcommand\labelitemii{\textasteriskcentered}
\renewcommand\labelitemiii{\textperiodcentered}
}

\begin{document}

\maketitle
\begin{center}
\fbox{
    \begin{minipage}[c]{15cm}
        Cette activité se destine aux élèves de 1ère après avoir étudié, le langage HTML et le CSS, ainsi que les bases du Javascript.
        
        
        Les compétences visées dans cette activité sont ambitieuses pour la plupart des élèves.
        Il faudra moduler les lignes de code de la partie javascript à donner  aux élèves en fonction du public visé.
    \end{minipage}
}
\end{center}


\newpage
\section{Présentation générale}

\begin{bclogo}{L'énoncé}
Le but du projet est d'obtenir une page web dynamique dont une vue est donnée ci-dessous.

L'utilisateur choisit un nombre de dés et un nombre de lancers.
En cliquant sur un bouton, la page devra donner la distribution en fréquence de la somme des dés et le diagramme en bâtons associés.

En cliquant à nouveau sur le bouton, les données statistiques \textbf{doivent être actualisées}.
\end{bclogo}

\begin{figure}[h]
    \centering
    \includegraphics[width=0.8\linewidth]{vue_page}
    \caption{}
    \label{fig:vuepage}
\end{figure}

\section{Contenu attendu}

\subsection{Code HTML}
Le fichier HTML devra contenir les éléments suivants :
\begin{enumerate}
    \item Le titre de la page.
    \item Une petite description de l'application.
    \item Un formulaire contenant :
    \begin{enumerate}
        \item un volet déroulant permettant de spécifier le nombre de dés à lancer (entre 1 et 7).
        \item une zone de texte permettant de spécifier le nombre de lancers.
    \end{enumerate}
    \item Un bouton qui permettra de lancer les dés.
    \item Un canvas de dimension 800px par 200px qui contiendra le diagramme des fréquences. 
    \item Un tableau à trois colonnes, une pour les différentes sommes obtenues, une pour les effectifs et une pour les fréquences.
    \item Deux conteneurs, un pour le tableau et un pour le reste de la page.
\end{enumerate}
\textbf{Note} Pour la suite (CSS et Javascript), vous aurez besoin d'accéder à la plupart (si ce n'est tous) les
éléments HTML que vous allez créer ici : assignez leur tous des identifiants !

\subsection{code CSS}
Le style, décrit dans un fichier style.css, doit contenir au minimum les propriétés suivantes :
\begin{itemize}
    \item Le tableau doit être centré sur la page.
    \item Les couleurs des lignes du tableaux des fréquences doivent alterner. 
    \item Vous n'avez droit qu'à une seule image pour représenter les dés (\textit{voir sprites}).
    \item La zone dans laquelle les dés sont représentés doit être mise en valeur.
\end{itemize}

\subsection{code Javascript}

Le fichier Script.js dont le code est recopié plus loin sera à compléter.
Il contient déjà quelques fonctions totalement définies :
\begin{itemize}
	\item de() permet d'obtenir le résultat d'un dé
	\item stat() permet de mettre à jour les variables somme et freq, qui sont des tableaux associatifs, contenant respectivement les effectifs et les fréquences des différentes sommes obtenues ;
\end{itemize}

Le code est à compléter pour contenir les fonctions nécessaires aux tâches à accomplir :

\begin{enumerate}
	\item Une fonction qui (re)trace le diagramme en bâton dans le canvas défini dans le code HTML.
	
	Voir par exemple ce \href{https://developer.mozilla.org/fr/docs/Web/API/CanvasRenderingContext2D}{lien} pour la gestion des canvas.
	\item Une fonction qui modifie le DOM en (re)créant les lignes du contenu du tableau de statistiques.
	\item La fonction exécutée lorsque l'on clique sur le bouton qui :
	\begin{enumerate}
		\item récupère le nombre de dés et le nombre de lancers ;
		\item réinitialise les sommes et les fréquences si le nombre de dés a changé ;
		\item effectue les lancers en mettant à jour les sommes ;
		\item met à jour les fréquences ;
		\item retrace le diagramme et refait le tableau ;
		\item affiche le résultat du dernier lancer (et cache les dés inutiles).
		
		Pour cela, utiliser des changements de classes avec la méthode setAttribute.
	\end{enumerate}
\end{enumerate}

\newpage

{\bfseries Contenu du fichier Script.js :}

\begin{minted}{js}
// Récupération des éléments utiles du document html
// ...

// Variables partagées par les différentes fonctions du script
var couleur = ["#70CCC3","#26998D", "#A6FFC4","#FF66A5","#CC70C4"]; // couleurs du diagramme
var sommes= {};
var freq = {};
/// ...

function de(){ // Donne le résultat d’un lancer de dé,
               // soit un nombre aléatoire entre 1 et 6
    return  Math.floor(Math.random() * 6) + 1 ;
  };

function stat(){ // Calcul les fréquences
    T = 0 ;
    for(i in sommes){
      T += sommes[i] ;
    } // T est l'effectif total
    for(i in sommes){
      freq[i] = sommes[i]/T ;
    }
  };

// Affiche un graphique
function graphe(){
    var canvas ... // element de la page dans laquelle se fait le dessin
    var parent = document.getElementById('main');
    canvas.width = parent.clientWidth - 20;
    var ctx = canvas.getContext('2d');
    // (Ré)initialisation du graphique
    // ...
    // Calcul de M, maximum des fréquences
    // ...
    index = 0;
    // Affichage des rectangles du diagramme en bâtons
    // Affichage de la somme associée
    // ...

};

// Affiche les résultats dans le tableau selon les fréquences
function aff(){
  // ...
};

// lance les dés et met à jour les données
function lance_des(){
    var S = 0; // somme des dés du lancer
    res = []; // garde la liste des résultats des dés pour l'affichage du dernier lancer
    for (i = 0 ; i< demax; i++ ){
      // lancer un dé, mise à jour des variables
      // ...
    };
    // mise à jour de la variable sommes
    // ...
};

// fonction appelée après avoir cliqué sur lancer
function jet_des(){
    ancien = demax;
    rep = parseInt(document.getElementById('rep').value);
    // récuperation de la valeur d'une liste 
    demax = parseInt(listenb.options[listenb.selectedIndex].value);

    // Vérifier si le nombre de dés a changé, et remettre à zéro les donnés si c’est le cas
    // ...
    // lancer les dés puis afficher le dernier résultat
    // ...
    // Calcule les statistiques, affiche le tableau puis le graphique
    // ...
};

// Association du bouton à la fonction
// ...
\end{minted}

\end{document}
